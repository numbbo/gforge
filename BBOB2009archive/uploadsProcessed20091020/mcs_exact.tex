% This is based from "sig-alternate.tex" V1.8 June 2007
% This file should be compiled with V2.3 of "sig-alternate.cls" June 2007
%
% ----------------------------------------------------------------------------------------------------------------
%% This .tex source is an example which *does* use
% the .bib file (from which the .bbl file % is produced).
% REMEMBER HOWEVER: After having produced the .bbl file,
% and prior to final submission, you *NEED* to 'insert'
% your .bbl file into your source .tex file so as to provide
% ONE 'self-contained' source file.
%
% Information on the sig-alternate class file and on the
% GECCO workshop paper format and submission can be found at these
% locations:
% http://www.acm.org/sigs/publications/proceedings-templates#aL2
% http://www.sheridanprinting.com/typedept/gecco3.htm
%
% ================= IF YOU HAVE QUESTIONS =======================
% Questions regarding the SIGS styles, SIGS policies and
% procedures, Conferences etc. should be sent to
% Adrienne Griscti (griscti@acm.org)
%
% Technical questions to bbob@lri.fr
% ===============================================================
%

\documentclass{sig-alternate}

\usepackage{graphicx}
\usepackage{rotating}
\pdfpagewidth=8.5in
\pdfpageheight=11in
\special{papersize=8.5in,11in}

    \renewcommand{\topfraction}{1}	% max fraction of floats at top
    \renewcommand{\bottomfraction}{1} % max fraction of floats at bottom
    %   Parameters for TEXT pages (not float pages):
    \setcounter{topnumber}{3}
    \setcounter{bottomnumber}{3}
    \setcounter{totalnumber}{3}     % 2 may work better
    \setcounter{dbltopnumber}{4}    % for 2-column pages
    \renewcommand{\dbltopfraction}{1}	% fit big float above 2-col. text
    \renewcommand{\textfraction}{0.0}	% allow minimal text w. figs
    %   Parameters for FLOAT pages (not text pages):
    \renewcommand{\floatpagefraction}{0.80}	% require fuller float pages
	% N.B.: floatpagefraction MUST be less than topfraction !!
    \renewcommand{\dblfloatpagefraction}{0.7}	% require fuller float pages

\newcommand{\DIM}{\ensuremath{\mathrm{DIM}}}
\newcommand{\ERT}{\ensuremath{\mathrm{ERT}}}
\newcommand{\nruns}{\ensuremath{\mathrm{Nruns}}}
\newcommand{\Dfb}{\ensuremath{\Delta f_{\mathrm{best}}}}
\newcommand{\Df}{\ensuremath{\Delta f}}
\newcommand{\nbFEs}{\ensuremath{\mathrm{\#FEs}}}
\newcommand{\fopt}{\ensuremath{f_\mathrm{opt}}}

\newcommand{\bbobdatapath}{ppdata/}
\graphicspath{{\bbobdatapath}}

\begin{document}
%
% --- Author Metadata here, no need to modify---
\conferenceinfo{GECCO'09,} {July 8--12, 2009, Montr\'eal Qu\'ebec, Canada.}
\CopyrightYear{2009}
\crdata{978-1-60558-505-5/09/07}
% --- End of Author Metadata, , no need to modify ---

\title{Benchmarking of MCS on the Noiseless Function
Testbed
% \titlenote{If needed}
}
%\subtitle{Draft version
%\titlenote{Camera-ready paper due April 17th.}}

%
% You need the command \numberofauthors to handle the 'placement
% and alignment' of the authors beneath the title.
%
% For aesthetic reasons, we recommend 'three authors at a time'
% i.e. three 'name/affiliation blocks' be placed beneath the title.
%
% NOTE: You are NOT restricted in how many 'rows' of
% "name/affiliations" may appear. We just ask that you restrict
% the number of 'columns' to three.
%
% Because of the available 'opening page real-estate'
% we ask you to refrain from putting more than six authors
% (two rows with three columns) beneath the article title.
% More than six makes the first-page appear very cluttered indeed.
%
% Use the \alignauthor commands to handle the names
% and affiliations for an 'aesthetic maximum' of six authors.
% Add names, affiliations, addresses for
% the seventh etc. author(s) as the argument for the
% \additionalauthors command.
% These 'additional authors' will be output/set for you
% without further effort on your part as the last section in
% the body of your article BEFORE References or any Appendices.

\numberofauthors{2}
%
\author{
% You can go ahead and credit any number of authors here,
% e.g. one 'row of three' or two rows (consisting of one row of three
% and a second row of one, two or three).
%
% The command \alignauthor (no curly braces needed) should
% precede each author name, affiliation/snail-mail address and
% e-mail address. Additionally, tag each line of
% affiliation/address with \affaddr, and tag the
% e-mail address with \email.
%
% 1st. author
\alignauthor
Waltraud Huyer\\
       \affaddr{Fakult\"at f\"ur Mathematik}\\
       \affaddr{Universit\"at Wien}\\
       \affaddr{Nordbergstra\ss e 15}\\
       \affaddr{1090 Wien}\\
       \affaddr{Austria}\\
       \email{Waltraud.Huyer@univie.ac.at}
%% 2nd. author
\alignauthor
Arnold Neumaier\\
       \affaddr{Fakult\"at f\"ur Mathematik}\\
       \affaddr{Universit\"at Wien}\\
       \affaddr{Nordbergstra\ss e 15}\\
       \affaddr{1090 Wien}\\
       \affaddr{Austria}\\
       \email{Arnold.Neumaier@univie.ac.at}
} %author

\maketitle
\begin{abstract}

Benchmarking results with the MCS algorithm for bound-constrained global optimization
on the noiseless BBOB 2009 testbed are described. 

\end{abstract}

% Add any ACM category that you feel is needed
\category{G.1.6}{Numerical Analysis}{Optimization}{Global Optimization,
Unconstrained Optimization}
\category{F.2.1}{Analysis of Algorithms and Problem Complexity}{Numerical Algorithms and Problems}

% Complete with anything that is needed
\terms{Algorithms}

% Complete with anything that is needed
\keywords{Benchmarking, Black-box optimization}

\section{Introduction}

Inspired by the {DIRECT} method by Jones et al.~\cite{jones}, the global
optimization algorithm MCS (multilevel coordinate search) \cite{mcs} was developed 
to minimize an objective function on a box $[u,v]$ with finite or infinite bounds.
The algorithm proceeds by splitting the search space into smaller boxes, and each
box contains a point whose function value is known. In the partitioning procedure
parts where low function values are expected to be found are preferred.
By starting a local search from certain good points, an improved result is 
obtained.

\section{Algorithm Presentation}

Like {DIRECT}, the MCS algorithm combines global search (splitting boxes with
large unexplored territory) and local search (splitting boxes with good function
values). The key to balancing global and local search is the multilevel approach.
As a rough measure of the number of times a box has been processed, a level
$s \in \{1,\dots,s_{\max}\}$ is assigned to each box, where boxes with level $s_{\max}$ 
are considered too small for further splitting. Whenever a box of level $s$
($0 < s < s_{\max}$) is split, its descendants get level $s+1$ or $\min(s+2,s_{\max})$.
After an initialization procedure, the algorithm proceeds by a series of sweeps
through the levels, i.e., it splits one box at each level, starting with the 
smallest non-empty level (i.e., with the largest boxes). We split along a single
coordinate in each step, and information gained from already sampled points
is used to determine the splitting coordinate as well as the position of the split.

MCS with local search tries to accelerate the convergence of the algorithm by
starting local searches from the points belonging to boxes of level $s_{\max}$
before putting them into the so-called shopping basket (containing `useful' points). 
The local search algorithm essentially consists of building a local quadratic
model by triple searches, then defining a promising search direction by minimizing 
the quadratic model on a suitable box and finally making a line search along this
direction.
 
The algorithm starts with a so-called initialization procedure producing an 
initial set of boxes. For each coordinate $i =1,\dots,n$, at least three values
$x_i^1 < x_i^2 < \dots < x_i^{L_i}$ in $[u_i,v_i]$, are needed, where $n$ denotes 
the dimension of the problem and $L_i \ge 3$. Moreover, the pointers $l_i \in 
\{1,\dots,L_i\}$ point to the initial point $x^0$, i.e., $x_i^0 = x_i^{l_i}$. The
values $x_i^j$, $j=1,\dots,L_i$, $l_i$, and $L_i$, $i=1,\dots,n$, constitute the
so-called initialization list.

The version of the software used can be downloaded from
\verb|http://www.mat.univie.ac.at/~neum/software/mcs/|.


\section{Experimental Procedure}
%

For all control variables in the algorithm meaningful default values can be 
chosen that work simultaneously for most problems. MCS essentially contains 
the following parameters: the number $s_{\max}$ of levels, a limit $nf_{\max}$ on the 
overall number of function calls, an additional stopping criterion, the 
initialization list, and a limit $nf_\text{local}$ on the number of function calls 
in a local search. We use the default value $s_{\max} = 5n+10$, and the additional 
stopping criterion is given by reaching a target function value $f_\text{target}$.

Five kinds of initialization lists are incorporated into the MCS software. The 
safeguarded version for infinite box bounds was not considered since all the box
bounds in our problems are finite, $u = (-5,\dots,-5)^T$ and $v = (5,\dots,5)^T$. 
The default initialization list for finite $u$ and $v$ consists of boundary points and
midpoint, with the midpoint as starting point, i.e., $L_i=3$, $l_i = 2$, $x_i^1 = u_i$,
$x_i^2 = \frac12(u_i+v_i)$, and $x_i^3 = v_i$, $i=1,\dots,n$. 
Another initialization list for finite bounds uses $x_i^1 = \frac56 u_i + \frac16 v_i$ and $x_i^3 = \frac16
u_i + \frac56 v_i$ instead of the boundaries (all other quantities are the same).
There is also an option to generate an initialization list with the aid of line
searches (described in detail Section 7.6 of \cite{mcs}). We call the
MCS algorithm with these three kinds of initialization lists MCS1, MCS2, and MCS3,
respectively. Finally, it is
possible to use a self-defined initialization list. After the initialization list 
has been chosen, MCS is purely deterministic, so the initialization list is the
only possibility to introduce a random element in MCS.

In each call to MCS, we use $nf_{\max} = 500\max(n,10)$ (i.e., $nf_{\max} = 5000$
for $n=2, 3, 5, 10$ and $nf_{\max} = 10000$ for $n=20$) and $nf_\text{local} = 
\text{round}(nf_{\max}/5)$, and $nf_{\max}$ might be slightly exceeded since the
algorithm does not contain a check whether $nf_{\max}$ has been reached after each
function call. Each trial consists of first applying the predefined initialization
lists MCS1, MCS2, and MCS3 to the problem and then using a self-defined 
initialization list with $L_i = 3$, $l_i = 2$, and the values $x_i^j$, $j=1,2,3$, 
drawn uniformly from $[u_i,v_i]$ for $i=1,\dots,n$ for at most 7 times for 
dimensions $n=2,3,5$ and at most 5 times for dimensions $n=10,20$ (in order to save
CPU time). I.e., each trial consists of at most 10 (or 8) attempts to solve the
problem with MCS, and each call to MCS does not use any results from the previous
calls. If the target function value $f_\text{target}$ is reached, the trial is
terminated and the subsequent calls to MCS are not made any more. So at most
$50000$ function calls (possibly a few more) are made in each trial for $n=2,3,5$
and $4000\max(n,10)$ for $n=10,20$.
Three trials are made for the 5 function instances of each function. Note that
if MCS1, 
MCS2, or MCS3 already solves the problem, the results are the same for the three 
trials since these choices of initialization lists do not contain a random element. 

\section{CPU timing experiment}

For the timing experiment according to \cite{hansen2009exp},
the experimental procedure described above was run on $f_8$
with at most 1000 function evaluations in each call to MCS and restarted until 
at least 30 seconds had passed. The timing experiment was carried out on an Intel
Pentium 4 3.00 GHz under Ubuntu 4.0.3 with MATLAB 7.4.0.336, where most of the
benchmarking tests were run. The results were
3.2, 2.0, 1.4, 1.1, 2.0, and 2.1 times $10^{-8}$ seconds per function evaluation
in dimensions 2, 3, 5, 10, 20, and 40, respectively.



%%%%%%%%%%%%%%%%%%%%%%%%%%%%%%%%%%%%%%%%%%%%%%%%%%%%%%%%%%%%%%%%%%%%%%%%%%%%%%%
\section{Results}
%%%%%%%%%%%%%%%%%%%%%%%%%%%%%%%%%%%%%%%%%%%%%%%%%%%%%%%%%%%%%%%%%%%%%%%%%%%%%%%
Results from experiments according to \cite{hansen2009exp} on the benchmark
functions given in \cite{wp200901,hansen2009fun} are
presented in Figures~\ref{fig:ERTgraphs} and \ref{fig:RLDs} and in
Table~\ref{tab:ERTs}. 
%%%%%%%%%%%%%%%%%%%%%%%%%%%%%%%%%%%%%%%%%%%%%%%%%%%%%%%%%%%%%%%%%%%%%%%%%%%%%%%
\begin{figure*}
\begin{tabular}{l@{\hspace*{-0.025\textwidth}}l@{\hspace*{-0.025\textwidth}}l@{\hspace*{-0.025\textwidth}}l}
\includegraphics[width=0.268\textwidth]{ppdata_f1}&
\includegraphics[width=0.268\textwidth]{ppdata_f2}&
\includegraphics[width=0.268\textwidth]{ppdata_f3}&
\includegraphics[width=0.268\textwidth]{ppdata_f4}\\[-2.2ex]
\includegraphics[width=0.268\textwidth]{ppdata_f5}&
\includegraphics[width=0.268\textwidth]{ppdata_f6}&
\includegraphics[width=0.268\textwidth]{ppdata_f7}&
\includegraphics[width=0.268\textwidth]{ppdata_f8}\\[-2.2ex]
\includegraphics[width=0.268\textwidth]{ppdata_f9}&
\includegraphics[width=0.268\textwidth]{ppdata_f10}&
\includegraphics[width=0.268\textwidth]{ppdata_f11}&
\includegraphics[width=0.268\textwidth]{ppdata_f12}\\[-2.2ex]
\includegraphics[width=0.268\textwidth]{ppdata_f13}&
\includegraphics[width=0.268\textwidth]{ppdata_f14}&
\includegraphics[width=0.268\textwidth]{ppdata_f15}&
\includegraphics[width=0.268\textwidth]{ppdata_f16}\\[-2.2ex]
\includegraphics[width=0.268\textwidth]{ppdata_f17}&
\includegraphics[width=0.268\textwidth]{ppdata_f18}&
\includegraphics[width=0.268\textwidth]{ppdata_f19}&
\includegraphics[width=0.268\textwidth]{ppdata_f20}\\[-2.2ex]
\includegraphics[width=0.268\textwidth]{ppdata_f21}&
\includegraphics[width=0.268\textwidth]{ppdata_f22}&
\includegraphics[width=0.268\textwidth]{ppdata_f23}&
\includegraphics[width=0.268\textwidth]{ppdata_f24}
\end{tabular}
 \caption{\label{fig:ERTgraphs}Expected Running Time (\ERT,
 {\Large$\bullet$}) to reach $\fopt+\Df$ and median number of function
 evaluations of successful trials ($+$), shown for $\Df =
 10, 1, 10^{-1}, 10^{-2}, 10^{-3}, 10^{-5}, 10^{-8}$ (the exponent is
 given in the legend of $f_1$ and $f_{24}$) versus dimension in
 log-log presentation.  The $\ERT(\Df)$ equals to $\nbFEs(\Df)$
 divided by the number of successful trials, where a trial is
 successful if $\fopt+\Df$ was surpassed during the trial.  The
 $\nbFEs(\Df)$ are the total number of function evaluations while
 $\fopt+\Df$ was not surpassed during the trial from all respective
 trials (successful and unsuccessful), and \fopt\ denotes the optimal
 function value. Crosses ($\times$) indicate the total number of
 function evaluations $\nbFEs(-\infty)$.  Numbers above ERT-symbols indicate the number of successful trials. Annotated numbers on the
 ordinate are decimal logarithms. Additional grid lines show linear
 and quadratic scaling. }
\end{figure*}
%%%%%%%%%%%%%%%%%%%%%%%%%%%%%%%%%%%%%%%%%%%%%%%%%%%%%%%%%%%%%%%%%%%%%%%%%%%%%%%
%%%%%%%%%%%%%%%%%%%%%%%%%%%%%%%%%%%%%%%%%%%%%%%%%%%%%%%%%%%%%%%%%%%%%%%%%%%%%%%
 \newcommand{\tablecaption}{Shown are, for a given target difference
 to the optimal function value \Df: the number of successful trials
 (\textbf{$\#$}); the expected running time to surpass $\fopt+\Df$
 (\ERT, see Figure~\ref{fig:ERTgraphs}); the \textbf{10\%}-tile and
 \textbf{90\%}-tile of the bootstrap distribution of \ERT; the average
 number of function evaluations in successful trials or, if none was
 successful, as last entry the median number of function evaluations
 to reach the best function value ($\text{RT}_\text{succ}$).  If
 $\fopt+\Df$ was never reached, figures in \textit{italics} denote the
 best achieved \Df-value of the median trial and the 10\% and
 90\%-tile trial.  Furthermore, N denotes the number of trials, and
 mFE denotes the maximum of number of function evaluations executed in
 one trial. See Figure~\ref{fig:ERTgraphs} for the names of
 functions. }
\begin{table*}
\centering
\input{\bbobdatapath ppdata_f1}
\input{\bbobdatapath ppdata_f2}
\input{\bbobdatapath ppdata_f3}
\input{\bbobdatapath ppdata_f4}
\input{\bbobdatapath ppdata_f5}
\input{\bbobdatapath ppdata_f6}
\input{\bbobdatapath ppdata_f7}
\input{\bbobdatapath ppdata_f8}
\input{\bbobdatapath ppdata_f9}
\input{\bbobdatapath ppdata_f10}
\input{\bbobdatapath ppdata_f11}
\input{\bbobdatapath ppdata_f12}
% %
%  \caption{\tablecaption
%  }
% \end{table*}
% %%%%%%%%%%%%%%%%%%%%%%%%%%%%%%%%%%%%%%%%%%%%%%%%%%%%%%%%%%%%%%%%%%%%%%%%%%%%%%%
% %%%%%%%%%%%%%%%%%%%%%%%%%%%%%%%%%%%%%%%%%%%%%%%%%%%%%%%%%%%%%%%%%%%%%%%%%%%%%%%
% \begin{table*}
% \centering
\input{\bbobdatapath ppdata_f13}
\input{\bbobdatapath ppdata_f14}
\input{\bbobdatapath ppdata_f15}
\input{\bbobdatapath ppdata_f16}
\input{\bbobdatapath ppdata_f17}
\input{\bbobdatapath ppdata_f18}
\input{\bbobdatapath ppdata_f19}
\input{\bbobdatapath ppdata_f20}
\input{\bbobdatapath ppdata_f21}
\input{\bbobdatapath ppdata_f22}
\input{\bbobdatapath ppdata_f23}
\input{\bbobdatapath ppdata_f24}\\
\caption[Table of ERTs]{\label{tab:ERTs}\tablecaption
}
\end{table*}
%%%%%%%%%%%%%%%%%%%%%%%%%%%%%%%%%%%%%%%%%%%%%%%%%%%%%%%%%%%%%%%%%%%%%%%%%%%%%%%
%%%%%%%%%%%%%%%%%%%%%%%%%%%%%%%%%%%%%%%%%%%%%%%%%%%%%%%%%%%%%%%%%%%%%%%%%%%%%%%
\newcommand{\rot}[2][2.5]{
  \hspace*{-3.5\baselineskip}%
  \begin{rotate}{90}\hspace{#1em}#2
  \end{rotate}}
\begin{figure*}
\begin{tabular}{l@{\hspace*{-0.025\textwidth}}l@{\hspace*{-0.00\textwidth}}|l@{\hspace*{-0.025\textwidth}}l}
\multicolumn{2}{c}{$D=5$} & \multicolumn{2}{c}{$D=20$}\\[-0.5ex]
\rot{all functions}
\includegraphics[width=0.268\textwidth]{pprldistr_dim05all} & 
\includegraphics[width=0.2362\textwidth,trim=2.40cm 0 0 0, clip]{ppfvdistr_dim05all} &
\includegraphics[width=0.268\textwidth]{pprldistr_dim20all} &
\includegraphics[width=0.2362\textwidth,trim=2.40cm 0 0 0, clip]{ppfvdistr_dim20all} \\[-2.3ex]
\rot{separable fcts}
\includegraphics[width=0.268\textwidth]{pprldistr_dim05separ} &
\includegraphics[width=0.2362\textwidth,trim=2.40cm 0 0 0, clip]{ppfvdistr_dim05separ} &
\includegraphics[width=0.268\textwidth]{pprldistr_dim20separ} &
\includegraphics[width=0.2362\textwidth,trim=2.40cm 0 0 0, clip]{ppfvdistr_dim20separ} \\[-2.3ex]
\rot[2]{moderate fcts}
\includegraphics[width=0.268\textwidth]{pprldistr_dim05lcond} &
\includegraphics[width=0.23628\textwidth,trim=2.40cm 0 0 0, clip]{ppfvdistr_dim05lcond} &
\includegraphics[width=0.268\textwidth]{pprldistr_dim20lcond} &
\includegraphics[width=0.2362\textwidth,trim=2.40cm 0 0 0, clip]{ppfvdistr_dim20lcond} \\[-2.3ex]
\rot[1.3]{ill-conditioned fcts}
\includegraphics[width=0.268\textwidth]{pprldistr_dim05hcond} &
\includegraphics[width=0.2362\textwidth,trim=2.40cm 0 0 0, clip]{ppfvdistr_dim05hcond} &
\includegraphics[width=0.268\textwidth]{pprldistr_dim20hcond} &
\includegraphics[width=0.2362\textwidth,trim=2.40cm 0 0 0, clip]{ppfvdistr_dim20hcond} \\[-2.3ex]
\rot[1.6]{multi-modal fcts}
\includegraphics[width=0.268\textwidth]{pprldistr_dim05multi} &
\includegraphics[width=0.2362\textwidth,trim=2.40cm 0 0 0, clip]{ppfvdistr_dim05multi} &
\includegraphics[width=0.268\textwidth]{pprldistr_dim20multi} &
\includegraphics[width=0.2362\textwidth,trim=2.40cm 0 0 0, clip]{ppfvdistr_dim20multi} \\[-2.3ex]
\rot[1.0]{weak structure fcts}
\includegraphics[width=0.268\textwidth]{pprldistr_dim05mult2} &
\includegraphics[width=0.2362\textwidth,trim=2.40cm 0 0 0, clip]{ppfvdistr_dim05mult2} &
\includegraphics[width=0.268\textwidth]{pprldistr_dim20mult2} &
\includegraphics[width=0.2362\textwidth,trim=2.40cm 0 0 0, clip]{ppfvdistr_dim20mult2}
\vspace*{-0.5ex}
\end{tabular}
 \caption{\label{fig:RLDs}Empirical cumulative distribution functions (ECDFs), plotting the fraction of trials versus running time (left) or \Df.  Left subplots: ECDF of the running time (number of function evaluations), divided by
 search space dimension $D$, to fall below $\fopt+\Df$ with
 $\Df=10^{k}$, where $k$ is the first value in the legend. Right subplots: ECDF of the best achieved \Df\ divided by $10^k$ (upper left
 lines in continuation of the left subplot), and best achieved \Df\
 divided by $10^{-8}$ for running times of $D, 10\,D,
 100\,D\dots$ function evaluations (from right
 to left cycling black-cyan-magenta).
 Top row: all results from all functions; second row: separable
 functions; third row: misc.\ moderate functions; fourth row:
 ill-conditioned functions; fifth row: multi-modal functions with
 adequate structure; last row: multi-modal functions with weak structure.
 The legends indicate the number of functions that were solved in at
 least one trial. FEvals denotes number of function evaluations,
 $D$ and \textsf{DIM} denote search space dimension, and \Df\ and \textsf{Df} denote the difference to the optimal function value. }
\end{figure*}
%%%%%%%%%%%%%%%%%%%%%%%%%%%%%%%%%%%%%%%%%%%%%%%%%%%%%%%%%%%%%%%%%%%%%%%%%%%%%%%
%%%%%%%%%%%%%%%%%%%%%%%%%%%%%%%%%%%%%%%%%%%%%%%%%%%%%%%%%%%%%%%%%%%%%%%%%%%%%%%

%
% The following two commands are all you need in the
% initial runs of your .tex file to
% produce the bibliography for the citations in your paper.
\bibliographystyle{abbrv}
\bibliography{bbob}  % bbob.bib is the name of the Bibliography in this case
% You must have a proper ".bib" file
%  and remember to run:
% latex bibtex latex latex
% to resolve all references
% to create the ~.bbl file.  Insert that ~.bbl file into
% the .tex source file and comment out
% the command \texttt{{\char'134}thebibliography}.
%
% ACM needs 'a single self-contained file'!
%
\end{document}
