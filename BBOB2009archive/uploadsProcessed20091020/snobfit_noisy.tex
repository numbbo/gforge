% This is based from "sig-alternate.tex" V1.8 June 2007
% This file should be compiled with V2.3 of "sig-alternate.cls" June 2007
%
% ----------------------------------------------------------------------------------------------------------------
%
% This .tex source is an example which *does* use
% the .bib file (from which the .bbl file % is produced).
% REMEMBER HOWEVER: After having produced the .bbl file,
% and prior to final submission, you *NEED* to 'insert'
% your .bbl file into your source .tex file so as to provide
% ONE 'self-contained' source file.
%
% Information on the sig-alternate class file and on the
% GECCO workshop paper format and submission can be found at these
% locations:
% http://www.acm.org/sigs/publications/proceedings-templates#aL2
% http://www.sheridanprinting.com/typedept/gecco3.htm
%
% ================= IF YOU HAVE QUESTIONS =======================
% Questions regarding the SIGS styles, SIGS policies and
% procedures, Conferences etc. should be sent to
% Adrienne Griscti (griscti@acm.org)
%
% Technical questions to bbob@lri.fr
% ===============================================================
%

\documentclass{sig-alternate}

\usepackage{graphicx}
\usepackage{rotating}
\pdfpagewidth=8.5in
\pdfpageheight=11in
\special{papersize=8.5in,11in}

    \renewcommand{\topfraction}{1}	% max fraction of floats at top
    \renewcommand{\bottomfraction}{1} % max fraction of floats at bottom
    %   Parameters for TEXT pages (not float pages):
    \setcounter{topnumber}{3}
    \setcounter{bottomnumber}{3}
    \setcounter{totalnumber}{3}     % 2 may work better
    \setcounter{dbltopnumber}{4}    % for 2-column pages
    \renewcommand{\dbltopfraction}{1}	% fit big float above 2-col. text
    \renewcommand{\textfraction}{0.0}	% allow minimal text w. figs
    %   Parameters for FLOAT pages (not text pages):
    \renewcommand{\floatpagefraction}{0.80}	% require fuller float pages
	% N.B.: floatpagefraction MUST be less than topfraction !!
    \renewcommand{\dblfloatpagefraction}{0.7}	% require fuller float pages

\newcommand{\DIM}{\ensuremath{\mathrm{DIM}}}
\newcommand{\ERT}{\ensuremath{\mathrm{ERT}}}
\newcommand{\nruns}{\ensuremath{\mathrm{Nruns}}}
\newcommand{\Dfb}{\ensuremath{\Delta f_{\mathrm{best}}}}
\newcommand{\Df}{\ensuremath{\Delta f}}
\newcommand{\nbFEs}{\ensuremath{\mathrm{\#FEs}}}
\newcommand{\fopt}{\ensuremath{f_\mathrm{opt}}}

\newcommand{\bbobdatapath}{ppdata/}
\hyphenation{quad-rat-ic}
\graphicspath{{\bbobdatapath}}

\begin{document}
%
% --- Author Metadata here, no need to modify---
\conferenceinfo{GECCO'09,} {July 8--12, 2009, Montr\'eal Qu\'ebec, Canada.}
\CopyrightYear{2009}
\crdata{978-1-60558-505-5/09/07}
% --- End of Author Metadata, , no need to modify ---

\title{Benchmarking of SNOBFIT on the Noisy Function
Testbed
% \titlenote{If needed}
}
%\subtitle{Draft version
%\titlenote{Camera-ready paper due April 17th.}}
%
% You need the command \numberofauthors to handle the 'placement
% and alignment' of the authors beneath the title.
%
% For aesthetic reasons, we recommend 'three authors at a time'
% i.e. three 'name/affiliation blocks' be placed beneath the title.
%
% NOTE: You are NOT restricted in how many 'rows' of
% "name/affiliations" may appear. We just ask that you restrict
% the number of 'columns' to three.
%
% Because of the available 'opening page real-estate'
% we ask you to refrain from putting more than six authors
% (two rows with three columns) beneath the article title.
% More than six makes the first-page appear very cluttered indeed.
%
% Use the \alignauthor commands to handle the names
% and affiliations for an 'aesthetic maximum' of six authors.
% Add names, affiliations, addresses for
% the seventh etc. author(s) as the argument for the
% \additionalauthors command.
% These 'additional authors' will be output/set for you
% without further effort on your part as the last section in
% the body of your article BEFORE References or any Appendices.

\numberofauthors{2} 
%
\author{
% You can go ahead and credit any number of authors here,
% e.g. one 'row of three' or two rows (consisting of one row of three
% and a second row of one, two or three).
%
% The command \alignauthor (no curly braces needed) should
% precede each author name, affiliation/snail-mail address and
% e-mail address. Additionally, tag each line of
% affiliation/address with \affaddr, and tag the
% e-mail address with \email.
%
% 1st. author
\alignauthor
Waltraud Huyer\\
       \affaddr{Fakult\"at f\"ur Mathematik}\\
       \affaddr{Universit\"at Wien}\\
       \affaddr{Nordbergstra\ss e 15}\\
       \affaddr{1090 Wien}\\
       \affaddr{Austria}\\
       \email{Waltraud.Huyer@univie.ac.at}
%% 2nd. author
\alignauthor
Arnold Neumaier\\
       \affaddr{Fakult\"at f\"ur Mathematik}\\
       \affaddr{Universit\"at Wien}\\
       \affaddr{Nordbergstra\ss e 15}\\
       \affaddr{1090 Wien}\\
       \affaddr{Austria}\\
       \email{Arnold.Neumaier@univie.ac.at}
} % author
%% There's nothing stopping you putting the seventh, eighth, etc.
%% author on the opening page (as the 'third row') but we ask,
%% for aesthetic reasons that you place these 'additional authors'
%% in the \additional authors block, viz.
%\additionalauthors{Additional authors: John Smith (The Th{\o}rv{\"a}ld Group,
%email: {\texttt{jsmith@affiliation.org}}) and Julius P.~Kumquat
%(The Kumquat Consortium, email: {\texttt{jpkumquat@consortium.net}}).}
%\date{30 July 1999}
%% Just remember to make sure that the TOTAL number of authors
%% is the number that will appear on the first page PLUS the
%% number that will appear in the \additionalauthors section.

\maketitle
\begin{abstract}

Benchmarking results with the SNOBFIT algorithm for noisy bound-constrained global
optimization on the noisy BBOB 2009 testbed are described.


\end{abstract}

% Add any ACM category that you feel is needed
\category{G.1.6}{Numerical Analysis}{Optimization}{Global Optimization,
Unconstrained Optimization}
\category{F.2.1}{Analysis of Algorithms and Problem Complexity}{Numerical Algorithms and Problems}

% Complete with anything that is needed
\terms{Algorithms}

% Complete with anything that is needed
\keywords{Benchmarking, Black-box optimization}

\section{Introduction}
%

The algorithm SNOBFIT (stable noisy optimization by branch and fit)
\cite{snobfit} for bound-constrained global
optimization of noisy functions
combines global and local search by
branching (i.e., splitting the search space $[u,v]$ into smaller boxes) and local 
fits. Based on function values already available, the algorithm builds internally
around each point local models of the function to minimize, and returns in each step 
a number of points whose evaluation is likely to improve these models or is expected
to give better function values. Surrogate functions are not interpolated but
fitted to a stochastic (linear or quadratic) model to take noisy function values
into account.

\section{Algorithm Presentation}
%

The optimization proceeds through a number of calls to SNOBFIT producing a 
user-specified number of new recommended evaluation points. SNOBFIT generates
points of different character belonging to five classes. Class 1 contains at
most one point, determined from the local quadratic model around the best point.
Points of classes 2 and 3 are alternative good points selected with a view to
their expected function value. The points in class 4 are points in regions
unexplored thus far, i.e., they are generated in large subboxes of the current
partition. Points of class 5 are only produced if the algorithm does not manage 
to reach the desired number of points by generating points of classes 1 to 4, 
for example, when there are not enough points yet available to build local
quadratic models, and they are chosen from a set of random points such that
their distances from the points already in the list are maximal.


In the so-called
initial call, a `resolution vector' $\Delta x >0$ is needed as an additional input. It
is assumed that the $i$th coordinate is measured in units $\Delta x_i$. The algorithm
only suggests evaluation points whose $i$th coordinate is an integral multiple of
$\Delta x_i$. In each call to SNOBFIT, a list $x^j$, $j=1,\dots,J$ of points, their
corresponding function values $f_j$, the uncertainties $\Delta f_j$ of the function
values, a natural number $n_\text{req}$, two vectors $u'$ and $v'$, and a real number
$p \in [0,1]$ (the desired fraction of points of class 4 among the points of 
classes 2 to 4, i.e., $p$ controls whether local or global search should be 
emphasized) are fed into the program. The program then returns $n_\text{req}$
(occasionally fewer) suggested evaluation points in the box $[u',v']$, their classes,
and the model function values at these points. The idea of the algorithm is that
these points and their function values are used as input for the next call to
SNOBFIT.

The version of the software used can be downloaded from
\verb|http://www.mat.univie.ac.at/~neum/software/snobfit/|.
 
\section{Experimental Procedure}
%
In our experiments, we always set $u'=u=(-5,\dots,-5)^T$ and $v'=v=(5,\dots,5)^T$
(i.e., the points are to be generated in the whole search space), $\Delta x = 
10^{-8}(v-u)$ and $p=0.1$, and we generate $n_\text{req} = n+6$ points in each
call to SNOBFIT, where $n$ is the dimension of the problem. These values are 
considered to be meaningful default values for the algorithm. If $f$ is the function
value at a point, the uncertainty $\Delta f$ is set to $0.03f$ for the functions
$f_{101}$ to $f_{106}$ (moderate noise) and $0.3f$ for the functions $f_{107}$ to
$f_{130}$ (severe noise). We start each trial with an initial call with $n+6$ 
points drawn uniformly from $[u,v]$ as input, and we proceed by so-called
continuation calls to SNOBFIT till 1000 function evaluations (including the ones
made in the initial call) have been reached. Afterwards we repeat this procedure
(at most) four times, but instead of sampling $n+6$ points for the initial calls,
we only sample $n+5$ points and in addition keep the best point from the previous
`iteration'. I.e., each trial consists of at most 5 attempts to solve the
problem with the SNOBFIT algorithm, and only the best point from the previous
attempt is reused. After each call to SNOBFIT, it is checked whether the
target function value $f_\text{target}$ has been reached, and in that case the
trial is terminated. So at most 5000 function calls (possibly a few more since 
reaching the number of permitted function values is only checked after each call
to SNOBFIT) are made in each trial.
Three trials are made for the 5 function instances of each function.

SNOBFIT uses the program MINQ for minimizing a quadratic model, which can be
downloaded from\\
 \verb|http://www.mat.univie.ac.at/~neum/software/minq/|.\\ 
Since the iteration limit in MINQ was frequently exceeded in higher dimensions, we
changed the line {\tt maxit=3*n} in {\tt minq.m} to {\tt maxit=10*n}.

\section{CPU timing experiment}

For the timing experiment according to \cite{hansen2009exp},
the experimental procedure described above was run on $f_8$
with at most 100 function evaluations in each SNOBFIT `iteration' and restarted 
until at least 30 seconds had passed. The uncertainties were set to $\Delta f
= \sqrt{\varepsilon}f$ (where $\varepsilon$ is the machine precision) since
$f_8$ is a noiseless function.
The timing experiment was carried out on an Intel Pentium 4 3.00 GHz under Ubuntu 4.0.3 
with MATLAB 7.4.0.336, where most of the benchmarking tests were run. 
The results were
8.3, 8.5, 8.8, 9.1, 10, and 7.9 times $10^{-1}$ seconds per function evaluation
in dimensions 2, 3, 5, 10, 20, and 40, respectively.


%%%%%%%%%%%%%%%%%%%%%%%%%%%%%%%%%%%%%%%%%%%%%%%%%%%%%%%%%%%%%%%%%%%%%%%%%%%%%%%
\section{Results}
%%%%%%%%%%%%%%%%%%%%%%%%%%%%%%%%%%%%%%%%%%%%%%%%%%%%%%%%%%%%%%%%%%%%%%%%%%%%%%%
Results from experiments according to \cite{hansen2009exp} on the
benchmarks functions given in \cite{wp200902,hansen2009noi} are
presented in Figures~\ref{fig:ERTgraphs} and \ref{fig:RLDs} and in
Tables~\ref{tab:ERTs1} and \ref{tab:ERTs2}.
%%%%%%%%%%%%%%%%%%%%%%%%%%%%%%%%%%%%%%%%%%%%%%%%%%%%%%%%%%%%%%%%%%%%%%%%%%%%%%%
%%%%%%%%%%%%%%%%%%%%%%%%%%%%%%%%%%%%%%%%%%%%%%%%%%%%%%%%%%%%%%%%%%%%%%%%%%%%%%
%% version with one page of figures %%%%%%%%%%%%%%%%%%%%%%%%%%%%%%%%%%%%%%%%%
%%%%%%%%%%%%%%%%%%%%%%%%%%%%%%%%%%%%%%%%%%%%%%%%%%%%%%%%%%%%%%%%%%%%%%%%%%%%%%
\begin{figure*}
\begin{tabular}{l@{\hspace*{-0.021\textwidth}}l@{\hspace*{-0.021\textwidth}}l@{\hspace*{-0.021\textwidth}}l@{\hspace*{-0.021\textwidth}}l}
\hspace*{-0.021\textwidth}\includegraphics[trim=5mm 0mm 0mm 0mm, clip, width=0.22\textwidth,]{ppdata_f101}&
\includegraphics[trim=5mm 0mm 0mm 0mm, clip, width=0.22\textwidth]{ppdata_f104}&
\includegraphics[trim=5mm 0mm 0mm 0mm, clip, width=0.22\textwidth]{ppdata_f107}&
\includegraphics[trim=5mm 0mm 0mm 0mm, clip, width=0.22\textwidth]{ppdata_f110}&
\includegraphics[trim=5mm 0mm 0mm 0mm, clip, width=0.22\textwidth]{ppdata_f113}\\
\hspace*{-0.021\textwidth}\includegraphics[trim=5mm 0mm 0mm 0mm, clip, width=0.22\textwidth]{ppdata_f102}&
\includegraphics[trim=5mm 0mm 0mm 0mm, clip, width=0.22\textwidth]{ppdata_f105}&
\includegraphics[trim=5mm 0mm 0mm 0mm, clip, width=0.22\textwidth]{ppdata_f108}&
\includegraphics[trim=5mm 0mm 0mm 0mm, clip, width=0.22\textwidth]{ppdata_f111}&
\includegraphics[trim=5mm 0mm 0mm 0mm, clip, width=0.22\textwidth]{ppdata_f114}\\
\hspace*{-0.021\textwidth}\includegraphics[trim=5mm 0mm 0mm 0mm, clip, width=0.22\textwidth]{ppdata_f103}&
\includegraphics[trim=5mm 0mm 0mm 0mm, clip, width=0.22\textwidth]{ppdata_f106}&
\includegraphics[trim=5mm 0mm 0mm 0mm, clip, width=0.22\textwidth]{ppdata_f109}&
\includegraphics[trim=5mm 0mm 0mm 0mm, clip, width=0.22\textwidth]{ppdata_f112}&
\includegraphics[trim=5mm 0mm 0mm 0mm, clip, width=0.22\textwidth]{ppdata_f115}\\
\hspace*{-0.021\textwidth}\includegraphics[trim=5mm 0mm 0mm 0mm, clip, width=0.22\textwidth]{ppdata_f116}&
\includegraphics[trim=5mm 0mm 0mm 0mm, clip, width=0.22\textwidth]{ppdata_f119}&
\includegraphics[trim=5mm 0mm 0mm 0mm, clip, width=0.22\textwidth]{ppdata_f122}&
\includegraphics[trim=5mm 0mm 0mm 0mm, clip, width=0.22\textwidth]{ppdata_f125}&
\includegraphics[trim=5mm 0mm 0mm 0mm, clip, width=0.22\textwidth]{ppdata_f128}\\
\hspace*{-0.021\textwidth}\includegraphics[trim=5mm 0mm 0mm 0mm, clip, width=0.22\textwidth]{ppdata_f117}&
\includegraphics[trim=5mm 0mm 0mm 0mm, clip, width=0.22\textwidth]{ppdata_f120}&
\includegraphics[trim=5mm 0mm 0mm 0mm, clip, width=0.22\textwidth]{ppdata_f123}&
\includegraphics[trim=5mm 0mm 0mm 0mm, clip, width=0.22\textwidth]{ppdata_f126}&
\includegraphics[trim=5mm 0mm 0mm 0mm, clip, width=0.22\textwidth]{ppdata_f129}\\
\hspace*{-0.021\textwidth}\includegraphics[trim=5mm 0mm 0mm 0mm, clip, width=0.22\textwidth]{ppdata_f118}&
\includegraphics[trim=5mm 0mm 0mm 0mm, clip, width=0.22\textwidth]{ppdata_f121}&
\includegraphics[trim=5mm 0mm 0mm 0mm, clip, width=0.22\textwidth]{ppdata_f124}&
\includegraphics[trim=5mm 0mm 0mm 0mm, clip, width=0.22\textwidth]{ppdata_f127}&
\includegraphics[trim=5mm 0mm 0mm 0mm, clip, width=0.22\textwidth]{ppdata_f130}\\
\end{tabular}
 \caption{\label{fig:ERTgraphs}Expected Running Time (\ERT,
 {\Large$\bullet$}) to reach $\fopt+\Df$ and median number of function
 evaluations of successful trials ($+$), shown for $\Df = 10, 1, 10^{-1},
 10^{-2}, 10^{-3}, 10^{-5}, 10^{-8}$ (the exponent is given in the legend of $f_{101}$ and $f_{130}$) versus dimension in log-log
 presentation.
 The $\ERT(\Df)$ equals to $\nbFEs(\Df)$
 divided by the number of successful trials, where a trial is
 successful if $\fopt+\Df$ was surpassed during the trial.
 The $\nbFEs(\Df)$ are the total number of function
 evaluations while $\fopt+\Df$ was not
 surpassed during the trial from all respective trials (successful and unsuccessful), and \fopt\ denotes the optimal
 function value. Crosses ($\times$) indicate the total number of function evaluations $\nbFEs(-\infty)$.
 Numbers above ERT-symbols indicate the number of successful trials. Annotated numbers on the ordinate
 are decimal logarithms. Additional grid lines show linear and
 quadratic scaling. }
\end{figure*}
%%%%%%%%%%%%%%%%%%%%%%%%%%%%%%%%%%%%%%%%%%%%%%%%%%%%%%%%%%%%%%%%%%%%%%%%%%%%%%%
%%%%%%%%%%%%%%%%%%%%%%%%%%%%%%%%%%%%%%%%%%%%%%%%%%%%%%%%%%%%%%%%%%%%%%%%%%%%%%%
 \newcommand{\tablecaption}[1]{Shown are, for functions #1 and for a
 given target difference to the optimal function value \Df: the number
 of successful trials (\textbf{$\#$}); the expected running time to
 surpass $\fopt+\Df$ (\ERT, see Figure~\ref{fig:ERTgraphs}); the
 \textbf{10\%}-tile and \textbf{90\%}-tile of the bootstrap
 distribution of \ERT; the average number of function evaluations in
 successful trials or, if none was successful, as last entry the median
 number of function evaluations to reach the best function value
 ($\text{RT}_\text{succ}$).  If $\fopt+\Df$ was never reached, figures in
 \textit{italics} denote the best achieved \Df-value of the median
 trial and the 10\% and 90\%-tile trial.  Furthermore, N denotes the
 number of trials, and mFE denotes the maximum of number of function
 evaluations executed in one trial. See Figure~\ref{fig:ERTgraphs} for
 the names of functions. }
\begin{table*}
\centering
\input{\bbobdatapath ppdata_f101}
\input{\bbobdatapath ppdata_f102}
\input{\bbobdatapath ppdata_f103}
\input{\bbobdatapath ppdata_f104}
\input{\bbobdatapath ppdata_f105}
\input{\bbobdatapath ppdata_f106}
\input{\bbobdatapath ppdata_f107}
\input{\bbobdatapath ppdata_f108}
\input{\bbobdatapath ppdata_f109}
\input{\bbobdatapath ppdata_f110}
\input{\bbobdatapath ppdata_f111}
\input{\bbobdatapath ppdata_f112}
\input{\bbobdatapath ppdata_f113}
\input{\bbobdatapath ppdata_f114}
\input{\bbobdatapath ppdata_f115}
\input{\bbobdatapath ppdata_f116}
\input{\bbobdatapath ppdata_f117}
\input{\bbobdatapath ppdata_f118}
\input{\bbobdatapath ppdata_f119}
\input{\bbobdatapath ppdata_f120}\\
\caption[Table of ERTs 1]{\label{tab:ERTs1}\tablecaption{$f_{101}$-$f_{120}$}
}
\end{table*}
%%%%%%%%%%%%%%%%%%%%%%%%%%%%%%%%%%%%%%%%%%%%%%%%%%%%%%%%%%%%%%%%%%%%%%%%%%%%%%%
%%%%%%%%%%%%%%%%%%%%%%%%%%%%%%%%%%%%%%%%%%%%%%%%%%%%%%%%%%%%%%%%%%%%%%%%%%%%%%%
\begin{table*}
\centering
\input{\bbobdatapath ppdata_f121}
\input{\bbobdatapath ppdata_f122}
\input{\bbobdatapath ppdata_f123}
\input{\bbobdatapath ppdata_f124}
\input{\bbobdatapath ppdata_f125}
\input{\bbobdatapath ppdata_f126}
\input{\bbobdatapath ppdata_f127}
\input{\bbobdatapath ppdata_f128}
\input{\bbobdatapath ppdata_f129}
\input{\bbobdatapath ppdata_f130}\\
\caption[Table of ERTs 2]{\label{tab:ERTs2}\tablecaption{$f_{121}$-$f_{130}$}
}
\end{table*}
%%%%%%%%%%%%%%%%%%%%%%%%%%%%%%%%%%%%%%%%%%%%%%%%%%%%%%%%%%%%%%%%%%%%%%%%%%%%%%%
%%%%%%%%%%%%%%%%%%%%%%%%%%%%%%%%%%%%%%%%%%%%%%%%%%%%%%%%%%%%%%%%%%%%%%%%%%%%%%%
\newcommand{\rot}[2][2.5]{
  \hspace*{-3.5\baselineskip}%
  \begin{rotate}{90}\hspace{#1em}#2
  \end{rotate}}
\begin{figure*}
\begin{tabular}{l@{\hspace*{-0.025\textwidth}}l@{\hspace*{-0.00\textwidth}}|l@{\hspace*{-0.025\textwidth}}l}
\multicolumn{2}{c}{$D=5$} & \multicolumn{2}{c}{$D=20$}\\
\rot{all functions}
\includegraphics[width=0.268\textwidth]{pprldistr_dim05all} & 
\includegraphics[width=0.2362\textwidth,trim=2.40cm 0 0 0, clip]{ppfvdistr_dim05all} &
\includegraphics[width=0.268\textwidth]{pprldistr_dim20all} &
\includegraphics[width=0.2362\textwidth,trim=2.40cm 0 0 0, clip]{ppfvdistr_dim20all} \\
\rot{moderate noise}
\includegraphics[width=0.268\textwidth]{pprldistr_dim05nzmod} & 
\includegraphics[width=0.2362\textwidth,trim=2.40cm 0 0 0, clip]{ppfvdistr_dim05nzmod} &
\includegraphics[width=0.268\textwidth]{pprldistr_dim20nzmod} &
\includegraphics[width=0.2362\textwidth,trim=2.40cm 0 0 0, clip]{ppfvdistr_dim20nzmod} \\
\rot{severe noise}
\includegraphics[width=0.268\textwidth]{pprldistr_dim05nzsev} &
\includegraphics[width=0.2362\textwidth,trim=2.40cm 0 0 0, clip]{ppfvdistr_dim05nzsev} &
\includegraphics[width=0.268\textwidth]{pprldistr_dim20nzsev} &
\includegraphics[width=0.2362\textwidth,trim=2.40cm 0 0 0, clip]{ppfvdistr_dim20nzsev} \\
\rot[0.5]{severe noise multimod.}
\includegraphics[width=0.268\textwidth]{pprldistr_dim05nzsmm} &
\includegraphics[width=0.2362\textwidth,trim=2.40cm 0 0 0, clip]{ppfvdistr_dim05nzsmm} &
\includegraphics[width=0.268\textwidth]{pprldistr_dim20nzsmm} &
\includegraphics[width=0.2362\textwidth,trim=2.40cm 0 0 0, clip]{ppfvdistr_dim20nzsmm}
\end{tabular}
 \caption{\label{fig:RLDs}Empirical cumulative distribution functions (ECDFs), plotting the fraction of trials versus running time (left) or \Df.  Left subplots: ECDF of the running time (number of function evaluations), divided by
 search space dimension $D$, to fall below $\fopt+\Df$ with
 $\Df=10^{k}$, where $k$ is the first value in the legend. Right subplots: ECDF of the best achieved \Df\ divided by $10^k$ (upper left
 lines in continuation of the left subplot), and best achieved \Df\
 divided by $10^{-8}$ for running times of $D, 10\,D,
 100\,D\dots$ function evaluations (from right
 to left cycling black-cyan-magenta).
 Top row: all results from all functions; second row: moderate noise
 functions; third row: severe noise functions; fourth row: severe noise
 and highly-multimodal functions.
 The legends indicate the number of functions that were solved in at
 least one trial. FEvals denotes number of function evaluations,
 $D$ and \textsf{DIM} denote search space dimension, and \Df\ and \textsf{Df} denote the difference to the optimal function value. }
\end{figure*}
%%%%%%%%%%%%%%%%%%%%%%%%%%%%%%%%%%%%%%%%%%%%%%%%%%%%%%%%%%%%%%%%%%%%%%%%%%%%%%%
%%%%%%%%%%%%%%%%%%%%%%%%%%%%%%%%%%%%%%%%%%%%%%%%%%%%%%%%%%%%%%%%%%%%%%%%%%%%%%%

%
% The following two commands are all you need in the
% initial runs of your .tex file to
% produce the bibliography for the citations in your paper.
\bibliographystyle{abbrv}
\bibliography{bbob}  % bbob.bib is the name of the Bibliography in this case
% You must have a proper ".bib" file
%  and remember to run:
% latex bibtex latex latex
% to resolve all references
% to create the ~.bbl file.  Insert that ~.bbl file into
% the .tex source file and comment out
% the command \texttt{{\char'134}thebibliography}.
%
% ACM needs 'a single self-contained file'!
%
\end{document}
