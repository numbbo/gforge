% This is based from "sig-alternate.tex" V1.8 June 2007
% This file should be compiled with V2.3 of "sig-alternate.cls" June 2007
%
% ----------------------------------------------------------------------------------------------------------------
%
% This .tex source is an example which *does* use
% the .bib file (from which the .bbl file % is produced).
% REMEMBER HOWEVER: After having produced the .bbl file,
% and prior to final submission, you *NEED* to 'insert'
% your .bbl file into your source .tex file so as to provide
% ONE 'self-contained' source file.
%
% Information on the sig-alternate class file and on the
% GECCO workshop paper format and submission can be found at these
% locations:
% http://www.acm.org/sigs/publications/proceedings-templates#aL2
% http://www.sheridanprinting.com/typedept/gecco3.htm
%
% ================= IF YOU HAVE QUESTIONS =======================
% Questions regarding the SIGS styles, SIGS policies and
% procedures, Conferences etc. should be sent to
% Adrienne Griscti (griscti@acm.org)
%
% Technical questions to bbob@lri.fr
% ===============================================================
%

\documentclass{sig-alternate}

\usepackage{graphicx}
\usepackage{rotating}
\usepackage{algorithm}
\pdfpagewidth=8.5in
\pdfpageheight=11in
\special{papersize=8.5in,11in}

    \renewcommand{\topfraction}{1}	% max fraction of floats at top
    \renewcommand{\bottomfraction}{1} % max fraction of floats at bottom
    %   Parameters for TEXT pages (not float pages):
    \setcounter{topnumber}{3}
    \setcounter{bottomnumber}{3}
    \setcounter{totalnumber}{3}     % 2 may work better
    \setcounter{dbltopnumber}{4}    % for 2-column pages
    \renewcommand{\dbltopfraction}{1}	% fit big float above 2-col. text
    \renewcommand{\textfraction}{0.0}	% allow minimal text w. figs
    %   Parameters for FLOAT pages (not text pages):
    \renewcommand{\floatpagefraction}{0.80}	% require fuller float pages
	% N.B.: floatpagefraction MUST be less than topfraction !!
    \renewcommand{\dblfloatpagefraction}{0.7}	% require fuller float pages

\newcommand{\DIM}{\ensuremath{\mathrm{DIM}}}
\newcommand{\ERT}{\ensuremath{\mathrm{ERT}}}
\newcommand{\nruns}{\ensuremath{\mathrm{Nruns}}}
\newcommand{\Dfb}{\ensuremath{\Delta f_{\mathrm{best}}}}
\newcommand{\Df}{\ensuremath{\Delta f}}
\newcommand{\nbFEs}{\ensuremath{\mathrm{\#FEs}}}
\newcommand{\fopt}{\ensuremath{f_\mathrm{opt}}}

\newcommand{\bbobdatapath}{ppdata/}
\graphicspath{{\bbobdatapath}}

\begin{document}
%
% --- Author Metadata here, no need to modify---
\conferenceinfo{GECCO'09,} {July 8--12, 2009, Montr\'eal Qu\'ebec, Canada.}
\CopyrightYear{2009}
\crdata{978-1-60558-505-5/09/07}
% --- End of Author Metadata, , no need to modify ---

\title{BBO-Benchmarking of GLOBAL method for Noisy Function
Testbed
% \titlenote{If needed}
}
%\subtitle{Draft version
%\titlenote{Camera-ready paper due April 17th.}}
%
% You need the command \numberofauthors to handle the 'placement
% and alignment' of the authors beneath the title.
%
% For aesthetic reasons, we recommend 'three authors at a time'
% i.e. three 'name/affiliation blocks' be placed beneath the title.
%
% NOTE: You are NOT restricted in how many 'rows' of
% "name/affiliations" may appear. We just ask that you restrict
% the number of 'columns' to three.
%
% Because of the available 'opening page real-estate'
% we ask you to refrain from putting more than six authors
% (two rows with three columns) beneath the article title.
% More than six makes the first-page appear very cluttered indeed.
%
% Use the \alignauthor commands to handle the names
% and affiliations for an 'aesthetic maximum' of six authors.
% Add names, affiliations, addresses for
% the seventh etc. author(s) as the argument for the
% \additionalauthors command.
% These 'additional authors' will be output/set for you
% without further effort on your part as the last section in
% the body of your article BEFORE References or any Appendices.

\numberofauthors{4} %  in this sample file, there are a *total*
% of EIGHT authors. SIX appear on the 'first-page' (for formatting
% reasons) and the remaining two appear in the \additionalauthors section.
%
\author{
% You can go ahead and credit any number of authors here,
% e.g. one 'row of three' or two rows (consisting of one row of three
% and a second row of one, two or three).
%
% The command \alignauthor (no curly braces needed) should
% precede each author name, affiliation/snail-mail address and
% e-mail address. Additionally, tag each line of
% affiliation/address with \affaddr, and tag the
% e-mail address with \email.
%
% 1st. author
\alignauthor
Mih\'{a}ly Csaba Mark\'{o}t\\%\titlenote{The secretary disavows any knowledge of this author's actions.}\\
       \affaddr{University of Wien, Faculty of Mathematics}\\
%       \affaddr{Faculty of Mathematics}\\
       \affaddr{Nordbergstra{\ss}e 15}\\
       \affaddr{1090 Wien, Austria}\\
       \email{Mihaly.Markot@univie.ac.at}
%% 2nd. author
\alignauthor
Arnold Neumaier\\%\titlenote{The secretary disavows any knowledge of this author's actions.}\\
       \affaddr{University of Wien, Faculty of Mathematics}\\
%       \affaddr{Faculty of Mathematics}\\
       \affaddr{Nordbergstra{\ss}e 15}\\
       \affaddr{1090 Wien, Austria}\\
       \email{Arnold.Neumaier@univie.ac.at}
\and
%% 3rd. author
\alignauthor
L\'{a}szl\'{o} P\'{a}l\\ %\titlenote{Dr.~Trovato insisted his name be first.}\\
       \affaddr{Sapientia - Hungarian University of Transylvania}\\
       \affaddr{530140 Miercurea-Ciuc, Piata Libertatii, Nr. 1, Romania}\\
%       \affaddr{W}\\
       \email{pallaszlo@sapientia.siculorum.ro}
%\and  % use '\and' if you need 'another row' of author names
%% 4th. author
\alignauthor
Tibor Csendes\\ %\titlenote{Dr.~Trovato insisted his name be first.}\\
       \affaddr{University of Szeged}\\
       \affaddr{6701 Szeged, P.O. Box 652, Hungary }\\
%       \affaddr{W}\\
       \email{csendes@inf.u-szeged.hu}
%% 5th. author
%\alignauthor Sean Fogarty\\
%       \affaddr{NASA Ames Research Center}\\
%       \affaddr{Moffett Field}\\
%       \affaddr{California 94035}\\
%       \email{fogartys@amesres.org}
%% 6th. author
%\alignauthor Charles Palmer\\
%       \affaddr{Palmer Research Laboratories}\\
%       \affaddr{8600 Datapoint Drive}\\
%       \affaddr{San Antonio, Texas 78229}\\
%       \email{cpalmer@prl.com}
} % author
%% There's nothing stopping you putting the seventh, eighth, etc.
%% author on the opening page (as the 'third row') but we ask,
%% for aesthetic reasons that you place these 'additional authors'
%% in the \additional authors block, viz.
%\additionalauthors{Additional authors: John Smith (The Th{\o}rv\"{a}ld Group,
%email: {\texttt{jsmith@affiliation.org}}) and Julius P.~Kumquat
%(The Kumquat Consortium, email: {\texttt{jpkumquat@consortium.net}}).}
%\date{30 July 1999}
%% Just remember to make sure that the TOTAL number of authors
%% is the number that will appear on the first page PLUS the
%% number that will appear in the \additionalauthors section.

\maketitle
\begin{abstract}
GLOBAL is a multistart type stochastic method for bound constrained global optimization problems. It's goal is to find all the local minima that are potentially global. For this reason it involves a combination of sampling, clustering, and local search.
\end{abstract}

% Add any ACM category that you feel is needed
\category{G.1.6}{Numerical Analysis}{Optimization}{Global Optimization,
Unconstrained Optimization}
\category{F.2.1}{Analysis of Algorithms and Problem Complexity}{Numerical Algorithms and Problems}

% Complete with anything that is needed
%\terms{Algorithms}

% Complete with anything that is needed
\keywords{Benchmarking, Black-box optimization, Clustering methods}

 \section{Introduction}

 The multistart clustering global optimization method called GLOBAL \cite{csendes1988}
has been introduced in the 1980s for bound constrained global optimization problems
with black-box type objective function. The algorithm is based on Boender's algorithm \cite{boender1982} and it's goal is to find all local minimizer points that are potentially global. The local search procedures used by GLOBAL was either a quasi-Newton procedure with the DFP update formula or a random walk type direct search
method called UNIRANDI \cite{jarvi1973}. The GLOBAL it was originally coded in Fortran and C language.

Based on the old GLOBAL method we introduced a new version \cite{csendes2008} coded in MATLAB.
The algorithm was carefully studied, and was modified in some places to
achieve better reliability and efficiency while allowing higher dimensional problems
to be solved. In the new version we use the quasi-Newton local search method with
the BFGS update instead of the earlier DFP. We also combined GLOBAL with other local search methods like the Nelder-Mead simplex method. All the three version (Fortran, C, MATLAB) of the algorithm is freely available  for academic and nonprofit purposes at {\tt www.inf.u-szeged.hu/$\sim$csendes/regist.php} (after registration and limited for
low dimensional problems).

In this paper, the algorithm is benchmarked on the noisy BBOB 2009
testbed  \cite{wp200902, hansen2009noi} according to the experimental design from \cite{hansen2009exp}.


\section{Algorithm Presentation}

The GLOBAL method has two phases: a global and a local one. The global phase consist of sampling and clustering, while the local phase is based on local searches. The local minimizers
will be found by means of a local search procedure, starting from appropriately
chosen points from the sample drawn uniformly within the set of feasibility. In
an effort to identify the region of attraction of a local minimum, the procedure
invokes a clustering procedure. The main algorithm steps of it are summarized in
Algorithm 1:

\begin{algorithm}

\caption{A concise description of the GLOBAL optimization algorithm}

\begin{description}
\item[Step 1:] Draw $N$ points with uniform distribution in $X$, and add them to the
current cumulative sample $C$. Construct the transformed sample $T$ by taking
the $\gamma$ percent of the points in $C$ with the lowest function value.

\item[Step 2:] Apply the clustering procedure to $T$ one by one. If all points of $T$ can be
assigned to an existing cluster, go to Step 4

\item[Step 3:] Apply the local search procedure to the points in $T$ not yet
clustered. Repeat Step 3 until every point has been assigned to a cluster.

\item[Step 4:]If a new local minimizer has been found, go to Step 1.

\item[Step 5:] Determine the smallest local minimum value found, and stop.
\end{description} \label{a1}
\end{algorithm}


%
 \section{Experimental Procedure}

GLOBAL has six parameters to set: the number of sample points, the number of best points selected, the stopping criterion parameter for the local search, the maximum number of function evaluations for local search, the maximum number of local minima, the used local method. All these parameters have a default value and usually it needs to change only the first three.

In all dimensions and functions we used 300 sample points, and 2 best points. In 2,3 and 5 dimension the local search tolerance was $10^{-8}$, the maximum number of function evaluations for local search was 5000 and the local search was the simplex method. In 10 and 20 dimensions with the 103,108,110,111,114,116,117,120,123,126,129 functions we used the BFGS local search with $10^{-9}$ tolerance and with 10000 maximum function evaluations. In the case of the remained functions we applied the previous settings with the simplex local search procedure.

The corresponding crafting effort is: $CrE_{10}=CrE_{20}=-(\frac{11}{30}\ln\frac{11}{30}+\frac{19}{30}\ln\frac{19}{30})=0.6572$

%
%%%%%%%%%%%%%%%%%%%%%%%%%%%%%%%%%%%%%%%%%%%%%%%%%%%%%%%%%%%%%%%%%%%%%%%%%%%%%%%
\section{Results}
%%%%%%%%%%%%%%%%%%%%%%%%%%%%%%%%%%%%%%%%%%%%%%%%%%%%%%%%%%%%%%%%%%%%%%%%%%%%%%%
Results from experiments according to \cite{hansen2009exp} on the
benchmarks functions given in \cite{wp200902,hansen2009noi} are
presented in Figures~\ref{fig:ERTgraphs} and \ref{fig:RLDs} and in
Tables~\ref{tab:ERTs1} and \ref{tab:ERTs2}.
%%%%%%%%%%%%%%%%%%%%%%%%%%%%%%%%%%%%%%%%%%%%%%%%%%%%%%%%%%%%%%%%%%%%%%%%%%%%%%%
%%%%%%%%%%%%%%%%%%%%%%%%%%%%%%%%%%%%%%%%%%%%%%%%%%%%%%%%%%%%%%%%%%%%%%%%%%%%%%
%% version with one page of figures %%%%%%%%%%%%%%%%%%%%%%%%%%%%%%%%%%%%%%%%%
%%%%%%%%%%%%%%%%%%%%%%%%%%%%%%%%%%%%%%%%%%%%%%%%%%%%%%%%%%%%%%%%%%%%%%%%%%%%%%
\begin{figure*}
\begin{tabular}{l@{\hspace*{-0.021\textwidth}}l@{\hspace*{-0.021\textwidth}}l@{\hspace*{-0.021\textwidth}}l@{\hspace*{-0.021\textwidth}}l}
\hspace*{-0.021\textwidth}\includegraphics[trim=5mm 0mm 0mm 0mm, clip, width=0.22\textwidth,]{ppdata_f101}&
\includegraphics[trim=5mm 0mm 0mm 0mm, clip, width=0.22\textwidth]{ppdata_f104}&
\includegraphics[trim=5mm 0mm 0mm 0mm, clip, width=0.22\textwidth]{ppdata_f107}&
\includegraphics[trim=5mm 0mm 0mm 0mm, clip, width=0.22\textwidth]{ppdata_f110}&
\includegraphics[trim=5mm 0mm 0mm 0mm, clip, width=0.22\textwidth]{ppdata_f113}\\
\hspace*{-0.021\textwidth}\includegraphics[trim=5mm 0mm 0mm 0mm, clip, width=0.22\textwidth]{ppdata_f102}&
\includegraphics[trim=5mm 0mm 0mm 0mm, clip, width=0.22\textwidth]{ppdata_f105}&
\includegraphics[trim=5mm 0mm 0mm 0mm, clip, width=0.22\textwidth]{ppdata_f108}&
\includegraphics[trim=5mm 0mm 0mm 0mm, clip, width=0.22\textwidth]{ppdata_f111}&
\includegraphics[trim=5mm 0mm 0mm 0mm, clip, width=0.22\textwidth]{ppdata_f114}\\
\hspace*{-0.021\textwidth}\includegraphics[trim=5mm 0mm 0mm 0mm, clip, width=0.22\textwidth]{ppdata_f103}&
\includegraphics[trim=5mm 0mm 0mm 0mm, clip, width=0.22\textwidth]{ppdata_f106}&
\includegraphics[trim=5mm 0mm 0mm 0mm, clip, width=0.22\textwidth]{ppdata_f109}&
\includegraphics[trim=5mm 0mm 0mm 0mm, clip, width=0.22\textwidth]{ppdata_f112}&
\includegraphics[trim=5mm 0mm 0mm 0mm, clip, width=0.22\textwidth]{ppdata_f115}\\
\hspace*{-0.021\textwidth}\includegraphics[trim=5mm 0mm 0mm 0mm, clip, width=0.22\textwidth]{ppdata_f116}&
\includegraphics[trim=5mm 0mm 0mm 0mm, clip, width=0.22\textwidth]{ppdata_f119}&
\includegraphics[trim=5mm 0mm 0mm 0mm, clip, width=0.22\textwidth]{ppdata_f122}&
\includegraphics[trim=5mm 0mm 0mm 0mm, clip, width=0.22\textwidth]{ppdata_f125}&
\includegraphics[trim=5mm 0mm 0mm 0mm, clip, width=0.22\textwidth]{ppdata_f128}\\
\hspace*{-0.021\textwidth}\includegraphics[trim=5mm 0mm 0mm 0mm, clip, width=0.22\textwidth]{ppdata_f117}&
\includegraphics[trim=5mm 0mm 0mm 0mm, clip, width=0.22\textwidth]{ppdata_f120}&
\includegraphics[trim=5mm 0mm 0mm 0mm, clip, width=0.22\textwidth]{ppdata_f123}&
\includegraphics[trim=5mm 0mm 0mm 0mm, clip, width=0.22\textwidth]{ppdata_f126}&
\includegraphics[trim=5mm 0mm 0mm 0mm, clip, width=0.22\textwidth]{ppdata_f129}\\
\hspace*{-0.021\textwidth}\includegraphics[trim=5mm 0mm 0mm 0mm, clip, width=0.22\textwidth]{ppdata_f118}&
\includegraphics[trim=5mm 0mm 0mm 0mm, clip, width=0.22\textwidth]{ppdata_f121}&
\includegraphics[trim=5mm 0mm 0mm 0mm, clip, width=0.22\textwidth]{ppdata_f124}&
\includegraphics[trim=5mm 0mm 0mm 0mm, clip, width=0.22\textwidth]{ppdata_f127}&
\includegraphics[trim=5mm 0mm 0mm 0mm, clip, width=0.22\textwidth]{ppdata_f130}\\
\end{tabular}
 \caption{\label{fig:ERTgraphs}Expected Running Time (\ERT,
 {\Large$\bullet$}) to reach $\fopt+\Df$ and median number of function
 evaluations of successful trials ($+$), shown for $\Df = 10, 1, 10^{-1},
 10^{-2}, 10^{-3}, 10^{-5}, 10^{-8}$ (the exponent is given in the legend of $f_{101}$ and $f_{130}$) versus dimension in log-log
 presentation.
 The $\ERT(\Df)$ equals to $\nbFEs(\Df)$
 divided by the number of successful trials, where a trial is
 successful if $\fopt+\Df$ was surpassed during the trial.
 The $\nbFEs(\Df)$ are the total number of function
 evaluations while $\fopt+\Df$ was not
 surpassed during the trial from all respective trials (successful and unsuccessful), and \fopt\ denotes the optimal
 function value. Crosses ($\times$) indicate the total number of function evaluations $\nbFEs(-\infty)$.
 Numbers above ERT-symbols indicate the number of successful trials. Annotated numbers on the ordinate
 are decimal logarithms. Additional grid lines show linear and
 quadratic scaling. }
\end{figure*}
%%%%%%%%%%%%%%%%%%%%%%%%%%%%%%%%%%%%%%%%%%%%%%%%%%%%%%%%%%%%%%%%%%%%%%%%%%%%%%%
%%%%%%%%%%%%%%%%%%%%%%%%%%%%%%%%%%%%%%%%%%%%%%%%%%%%%%%%%%%%%%%%%%%%%%%%%%%%%%%
 \newcommand{\tablecaption}[1]{Shown are, for functions #1 and for a
 given target difference to the optimal function value \Df: the number
 of successful trials (\textbf{$\#$}); the expected running time to
 surpass $\fopt+\Df$ (\ERT, see Figure~\ref{fig:ERTgraphs}); the
 \textbf{10\%}-tile and \textbf{90\%}-tile of the bootstrap
 distribution of \ERT; the average number of function evaluations in
 successful trials or, if none was successful, as last entry the median
 number of function evaluations to reach the best function value
 ($\text{RT}_\text{succ}$).  If $\fopt+\Df$ was never reached, figures in
 \textit{italics} denote the best achieved \Df-value of the median
 trial and the 10\% and 90\%-tile trial.  Furthermore, N denotes the
 number of trials, and mFE denotes the maximum of number of function
 evaluations executed in one trial. See Figure~\ref{fig:ERTgraphs} for
 the names of functions. }
\begin{table*}
\centering
\input{\bbobdatapath ppdata_f101}
\input{\bbobdatapath ppdata_f102}
\input{\bbobdatapath ppdata_f103}
\input{\bbobdatapath ppdata_f104}
\input{\bbobdatapath ppdata_f105}
\input{\bbobdatapath ppdata_f106}
\input{\bbobdatapath ppdata_f107}
\input{\bbobdatapath ppdata_f108}
\input{\bbobdatapath ppdata_f109}
\input{\bbobdatapath ppdata_f110}
\input{\bbobdatapath ppdata_f111}
\input{\bbobdatapath ppdata_f112}
\input{\bbobdatapath ppdata_f113}
\input{\bbobdatapath ppdata_f114}
\input{\bbobdatapath ppdata_f115}
\input{\bbobdatapath ppdata_f116}
\input{\bbobdatapath ppdata_f117}
\input{\bbobdatapath ppdata_f118}
\input{\bbobdatapath ppdata_f119}
\input{\bbobdatapath ppdata_f120}\\
\caption[Table of ERTs 1]{\label{tab:ERTs1}\tablecaption{$f_{101}$-$f_{120}$}
}
\end{table*}
%%%%%%%%%%%%%%%%%%%%%%%%%%%%%%%%%%%%%%%%%%%%%%%%%%%%%%%%%%%%%%%%%%%%%%%%%%%%%%%
%%%%%%%%%%%%%%%%%%%%%%%%%%%%%%%%%%%%%%%%%%%%%%%%%%%%%%%%%%%%%%%%%%%%%%%%%%%%%%%
\begin{table*}
\centering
\input{\bbobdatapath ppdata_f121}
\input{\bbobdatapath ppdata_f122}
\input{\bbobdatapath ppdata_f123}
\input{\bbobdatapath ppdata_f124}
\input{\bbobdatapath ppdata_f125}
\input{\bbobdatapath ppdata_f126}
\input{\bbobdatapath ppdata_f127}
\input{\bbobdatapath ppdata_f128}
\input{\bbobdatapath ppdata_f129}
\input{\bbobdatapath ppdata_f130}\\
\caption[Table of ERTs 2]{\label{tab:ERTs2}\tablecaption{$f_{121}$-$f_{130}$}
}
\end{table*}
%%%%%%%%%%%%%%%%%%%%%%%%%%%%%%%%%%%%%%%%%%%%%%%%%%%%%%%%%%%%%%%%%%%%%%%%%%%%%%%
%%%%%%%%%%%%%%%%%%%%%%%%%%%%%%%%%%%%%%%%%%%%%%%%%%%%%%%%%%%%%%%%%%%%%%%%%%%%%%%
\newcommand{\rot}[2][2.5]{
  \hspace*{-3.5\baselineskip}%
  \begin{rotate}{90}\hspace{#1em}#2
  \end{rotate}}
\begin{figure*}
\begin{tabular}{l@{\hspace*{-0.025\textwidth}}l@{\hspace*{-0.00\textwidth}}|l@{\hspace*{-0.025\textwidth}}l}
\multicolumn{2}{c}{$D=5$} & \multicolumn{2}{c}{$D=20$}\\
\rot{all functions}
\includegraphics[width=0.268\textwidth]{pprldistr_dim05all} &
\includegraphics[width=0.2362\textwidth,trim=2.40cm 0 0 0, clip]{ppfvdistr_dim05all} &
\includegraphics[width=0.268\textwidth]{pprldistr_dim20all} &
\includegraphics[width=0.2362\textwidth,trim=2.40cm 0 0 0, clip]{ppfvdistr_dim20all} \\
\rot{moderate noise}
\includegraphics[width=0.268\textwidth]{pprldistr_dim05nzmod} &
\includegraphics[width=0.2362\textwidth,trim=2.40cm 0 0 0, clip]{ppfvdistr_dim05nzmod} &
\includegraphics[width=0.268\textwidth]{pprldistr_dim20nzmod} &
\includegraphics[width=0.2362\textwidth,trim=2.40cm 0 0 0, clip]{ppfvdistr_dim20nzmod} \\
\rot{severe noise}
\includegraphics[width=0.268\textwidth]{pprldistr_dim05nzsev} &
\includegraphics[width=0.2362\textwidth,trim=2.40cm 0 0 0, clip]{ppfvdistr_dim05nzsev} &
\includegraphics[width=0.268\textwidth]{pprldistr_dim20nzsev} &
\includegraphics[width=0.2362\textwidth,trim=2.40cm 0 0 0, clip]{ppfvdistr_dim20nzsev} \\
\rot[0.5]{severe noise multimod.}
\includegraphics[width=0.268\textwidth]{pprldistr_dim05nzsmm} &
\includegraphics[width=0.2362\textwidth,trim=2.40cm 0 0 0, clip]{ppfvdistr_dim05nzsmm} &
\includegraphics[width=0.268\textwidth]{pprldistr_dim20nzsmm} &
\includegraphics[width=0.2362\textwidth,trim=2.40cm 0 0 0, clip]{ppfvdistr_dim20nzsmm}
\end{tabular}
 \caption{\label{fig:RLDs}Empirical cumulative distribution functions (ECDFs), plotting the fraction of trials versus running time (left) or \Df.  Left subplots: ECDF of the running time (number of function evaluations), divided by
 search space dimension $D$, to fall below $\fopt+\Df$ with
 $\Df=10^{k}$, where $k$ is the first value in the legend. Right subplots: ECDF of the best achieved \Df\ divided by $10^k$ (upper left
 lines in continuation of the left subplot), and best achieved \Df\
 divided by $10^{-8}$ for running times of $D, 10\,D,
 100\,D\dots$ function evaluations (from right
 to left cycling black-cyan-magenta).
 Top row: all results from all functions; second row: moderate noise
 functions; third row: severe noise functions; fourth row: severe noise
 and highly-multimodal functions.
 The legends indicate the number of functions that were solved in at
 least one trial. FEvals denotes number of function evaluations,
 $D$ and \textsf{DIM} denote search space dimension, and \Df\ and \textsf{Df} denote the difference to the optimal function value. }
\end{figure*}
%%%%%%%%%%%%%%%%%%%%%%%%%%%%%%%%%%%%%%%%%%%%%%%%%%%%%%%%%%%%%%%%%%%%%%%%%%%%%%%
%%%%%%%%%%%%%%%%%%%%%%%%%%%%%%%%%%%%%%%%%%%%%%%%%%%%%%%%%%%%%%%%%%%%%%%%%%%%%%%

%
% The following two commands are all you need in the
% initial runs of your .tex file to
% produce the bibliography for the citations in your paper.
\bibliographystyle{abbrv}
\bibliography{bbob}  % bbob.bib is the name of the Bibliography in this case
% You must have a proper ".bib" file
%  and remember to run:
% latex bibtex latex latex
% to resolve all references
% to create the ~.bbl file.  Insert that ~.bbl file into
% the .tex source file and comment out
% the command \texttt{{\char'134}thebibliography}.
%
% ACM needs 'a single self-contained file'!
%
\end{document}
